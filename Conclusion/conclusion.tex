\par
Tout au long du développement, nous nous sommes efforcés de produire des applications répondant concrètement aux demandes de l'institut Montéclair, tout en gardant à l'esprit que les besoins de l'institut pourraient évoluer. Par conséquent, nos applications ne se veulent pas des versions définitives. En dehors du travail qui reste à effectuer pour les rendre totalement utilisables au quotidien, nous avons tenté de les rendre ouvertes à l'extension pour des versions futures avec peut-être d'autres types de QR Codes, d'autres types de contenus ou d'autres méthodes d'optimisation de la quantité de données contenues dans les QR Codes.\\
Cette volonté s'est traduite par la réalisation d'une architecture des classes souple et non spécifique aux besoins qui nous ont été transmis. Nous avons en outre fait en sorte que la compression des données, qui est la partie la plus susceptible d'être modifiée, soit modulable tout en restant facilement compatible avec les QR Codes que nous générons actuellement.\\

\par
Il reste toutefois une dernière phase de développement avant de pouvoir livrer un couple d'applications totalement fonctionnelles. Les modifications restantes concernent principalement l'interface graphique de l'application de bureau, afin qu'elle soit plus ergonomique avec un système de liste de QR Codes plutôt que d'onglets, un système de glisser-déposer de fichiers provenant de Google Drive, et l'implémentation de la création de QR Codes de type ensemble, permettant de télécharger un ensemble de fichiers audio sans les jouer.
\par
Elles concernent également la complétion de l'interprétation des QR Codes par l'application mobile, afin de gérer dynamiquement le téléchargement de fichiers à la lecture, et l'ajout de messages vocaux à destination de l'utilisateur.\\

\par
Lorsque cette phase de finalisation sera terminée, un nouveau projet pourra débuter à partir de là où nous l'avons laissé, comme nous l'avons fait à partir de l'application mobile initiale. En plus de l'implémentation d'éventuelles nouvelles demandes de l'institut Montéclair, un travail complexe mais intéressant pourra être effectué pour compresser le contenu des QR Codes de manière plus efficace, en compressant le texte contenu dans les QR Codes à partir d'un dictionnaire fixe externe. Une solution de transformation du texte en son directement sur l'application de bureau pourrait également permettre de créer des QR Codes contenant des textes de grande taille.