\par
À partir des contraintes définies dans les cahier des charges (voir \ref{objectifs}), il a fallu définir précisément quels types de QR Codes allaient être créés. Une première version de notre architecture collait aux besoins de l'Institut Montéclair qui nous ont été transmis. Elle comprenait des QR Codes pour les cartes géographiques, des QR Codes pour les pages d'un livre et des QR Codes pour les pages de couverture des livres. Mais cette architecture s'est avérée trop spécifique et empêchait les transcripteurs de l'institut de créer des QR Codes pour d'autres supports.\\

\par
Nous avons donc créé une seconde version qui respectait les contraintes transmises par l'institut tout en étant la plus généraliste possible, afin d'être utilisable dans tous les contextes compatibles avec les contraintes transmises. Cette architecture est composée de QR Codes contenant un ensemble de textes et de fichiers et pouvant être organisés en familles afin d'être lus dans un ordre prédéfini (les QR Codes atomiques), et de QR Codes contenant un ensemble de liens vers des sons devant être téléchargés sans être joués (les QR Codes ensembles).