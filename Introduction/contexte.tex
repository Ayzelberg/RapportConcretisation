
\par
Dans le cadre du module Concrétisation Disciplinaire de notre première année de Master 1 mention Informatique à l'Université d'Angers, nous avons développé une application de bureau et poursuivi le développement d'une application mobile.

\par
Ces applications répondent à une demande de l'Institut Montéclair à Angers, dans le but d'apporter une dimension ludique à l'enseignement pour les mal-voyants et non-voyants inscrits à l'institut.

\par
Ce projet a été encadré par Corentin \bsc{Talarmain} et Thomas \bsc{Calatayud}, deux étudiants en deuxième année de Master, dans le cadre de leur module Gestion de Projet.\\

\par
L'objectif du projet est de fournir la possibilité aux transcripteurs de l'institut un moyen de générer des QR Codes contenant du texte et des sons et pouvant être interprétés par une application mobile. Une première version de l'application mobile avait déjà été développée par Corentin \bsc{Talarmain} lors de son TER\footnote{Travail Encadré de Recherche} de fin de première année de Master. Celle-ci fournissait déjà la possibilité de lire avec une voix de synthèse du texte contenu dans des QR Codes.\\

