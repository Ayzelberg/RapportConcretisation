\par
Les transcripteurs souhaitaient néanmoins étendre les fonctionnalités de l'application mobile initiale, qui présentait quelques inconvénients limitant son utilisation. Ne présentant que la possibilité de lire des QR Codes, elle imposait le passage par des sites internet pour en générer. On ne pouvait ainsi pas contrôler la taille maximale des QR Codes générés, parmi lesquels certains étaient trop gros pour être détectables par l'application. En outre, elle ne fournissait pas la possibilité de lire des sons à partir de données contenues dans un QR Code.\\

\par
Les transcripteurs de l'institut nous ont également fourni deux demandes précises autour desquelles nous avons structuré notre projet. La première demande était de pouvoir utiliser les QR Codes comme légendes de cartes de géographie pour des lycéens. Cela implique de pouvoir gérer la lecture de QR Codes adjacents dans un ordre prédéfini, peu importe l'ordre de lecture effectif.
\par
L'autre demande était de pouvoir utiliser des QR Codes pour lire des sons dans les pages d'un livre pour enfants, sans avoir nécessairement accès à internet au moment de la lecture. Pour cela, un ou plusieurs QR Codes sur la page de couverture doivent permettre le téléchargement de tous les sons contenus dans le livre.
\par
En outre, un espace central contenant du texte en braille devait pouvoir être inséré au centre des QR Codes, pour permettre aux non-voyants de les localiser, ces caractères étant imprimés avec une encre spéciale gonflant à température élevée.