\par
Nous avons structuré notre projet pour répondre aux deux demandes principales qui nous ont été faites. La première demande était de pouvoir utiliser les QR Codes comme légendes de cartes de géographie. Cela implique de pouvoir gérer la lecture de QR Codes adjacents dans un ordre prédéfini, qu'importe l'ordre de lecture effectif (LIEN FAMILLES).\\

\par
L'autre demande était de pouvoir utiliser des QR Codes pour lire des sons dans les pages d'un livre, sans avoir nécessairement accès à internet au moment de la lecture. Pour cela, un ou plusieurs QR Codes sur la page de couverture doivent permettre le téléchargement de tous les sons contenus dans le livre (LIEN ENSEMBLE).\\

\par
De plus, un espace central contenant du texte en braille devait pouvoir être inséré au centre des QR Codes, pour permettre aux non-voyants de les localiser (LIEN GENERATION QR).