\par
Les QR Codes devant stocker plus d'informations que dans la version initiale de l'application mobile, il nous a fallu définir une structure de données pour les représenter. Cette représentation s'est construite parallèlement à l'élaboration des types de QR Codes (LIEN TYPES QR) et à celle du Modèle (LIEN MODÈLE), afin d'assurer la cohérence de l'ensemble.
\par
Notre choix s'est porté sur une structure XML\footnote{Hypertext Markup Language}. Cette structure a montré ses faiblesses par la suite (voir REF CHOIX XML), mais avait déjà crée trop de dépendances pour être modifiée.\\

\par
Afin de minimiser la quantité de données stockées dans le QR Code, seules les données indispensables à l'interprétation du QR Code par l'application mobile y sont stockées. Les données annexes nécessaires au chargement d'un QR Code par l'application de bureau sont stockées dans les métadonnées de l'image (REF CONTROLLEUR/METADONNÉES). On y trouve les données relatives au nom et aux couleurs du QR Code, ou encore au nom des fichiers contenus.
\par
La structure fait apparaître clairement la dichotomie entre les données stockées dans le QR Code (noeud \noeud{donneesUtilisateur}) et les données stockées dans les metadonnées de l'image (noeud \noeud{metadonnees}).\\

\par
Le type de QR Code (atomique ou ensemble) est indiqué par un attribut dans le noeud \noeud{donneesUtilisateur}. Ce noeud contient un noeud \noeud{contenu} contenant lui-même un ensemble de textes (\noeud{texte}) et de fichiers (\noeud{fichier}).
\par
L'appartenance à une famille (REF TYPES QR) de QR Codes est indiquée par la présence d'un noeud \noeud{famille} ayant pour attributs le nom de la famille et la place du QR Code.\\

\par
Un exemple complet de représentation XML d'un QR Code est visible dans la figure ci-dessous.

\begin{figure}[!h]
\begin{adjustbox}{minipage=\textwidth,bgcolor={RGB}{240 240 240}}

\lstset{language=XML}

\begin{lstlisting}

<qrcode>
  <donneesutilisateur type="atomique">
    <contenu>
      <texte>champ1</texte>
      <fichier url="URLFICHIER"></fichier>
      <texte>champ2</texte>
    </contenu>
    <famille nom="famille" ordre="4"></famille>
  </donneesutilisateur>
  <metadonnees>
    <fichiers>
      <fichier url="URLFICHIER" nom="fichier1"></fichier>
    </fichiers>
    <colorqrcode color="#085a0c"></colorqrcode>
    <textebraille texte="QR"></textebraille>
    <colorbraille color="#f11313"></colorbraille>
  </metadonnees>
</qrcode>

 
\end{lstlisting}

\end{adjustbox}
\caption{Représentation d'un QR Code}

\end{figure}\textbf{}