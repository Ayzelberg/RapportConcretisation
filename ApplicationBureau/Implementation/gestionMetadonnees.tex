\par
Dans l'objectif de minimiser la quantité de données stockées dans le QR Code, toutes les données qui ne sont pas nécessaires à l'application mobile mais qui ont leur intérêt dans le chargement de QR Codes déjà enregistrés par l'application de bureau sont stockées dans les métadonnées de l'image du QR Code (REF Représentation des données). De plus, les images de type sauvegarde de famille (REF chargement QRCode) contiennent l'intégralité de la représentation XML des QR Codes la formant dans les métadonnées.\\
\par
Pour insérer des métadonnées dans les images générées, nous utilisons le module Node.js piexifjs dans le processus suivant. Le QR Code est d'abord généré dans un canvas HTML 5, à partir duquel on va générer une image JPEG contenant dans les métadonnées le noeud racine de notre structure XML. Elles sont stockées dans le noeud XMLPacket des métadonnées EXIF. Nous avons choisi ce noeud pour éviter la perte d'information des caractères spécifiques au français, car il peut contenir des données binaires et pas seulement des caractères ASCII comme la plupart des noeuds EXIF. Les métadonnées EXIF n'existent pas dans les fichiers PNG, et nous n'avons pas trouvé de bibliothèque permettant d'insérer un autre type de métadonnées, c'est la raison pour laquelle nous générons des images en JPG.