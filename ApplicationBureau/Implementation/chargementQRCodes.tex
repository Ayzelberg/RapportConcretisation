\par
Dans le but de créer une application utilisable de façon concrète et régulière, nous avons créé des mécanismes permettant de modifier des QR Codes déjà enregistrés. Pour les QR Codes n'étant pas liés à d'autres (n'appartenant pas à une famille), il suffit d'enregistrer l'image représentant le QR Code en insérant dans les métadonnées le nœud racine de notre structure XML (REF REPRESENTATION DONNEES). L'utilisation du nœud XML \noeud{metadonnees} est par ailleurs trompeuse car l'intégralité de la structure XML est stockée dans les métadonnées de l'image. Nous avons choisi de procéder ainsi car nous n'avons pas trouvé de librairie Javascript permettant de scanner facilement un QR Code à partir d'un fichier image, et la quantité de données insérées dans les métadonnées de l'image est négligeable par rapport à la taille des données représentant l'image.\\
\par
Nous avons élaboré un mécanisme plus complexe en ce qui concerne l'enregistrement de QR Codes appartenant à une même famille. En effet, nous souhaitions toujours pouvoir enregistrer ces QR Codes séparément afin qu'ils puissent être imprimés, mais nous voulions empêcher la modification de l'un des QR Codes appartenant à une famille sans mettre à jour tous les autres. Pour cela, nous avons élaboré une façon d'enregistrer tous les QR Codes d'une famille dans un fichier unique, en plus des images représentant chacun des QR Codes. Nous avons choisi d'enregistrer ce fichier comme une image (REF GENERATION) contenant la représentation XML de tous les QR Codes de la famille dans ses métadonnées, et affichant certaines informations concernant la famille sous forme de texte imprimé sur l'image. Les informations imprimées sont le nom de la famille, le nombre de QR Codes contenus et la date de création. Cette image possède en outre un fond contenant quatre QR Codes colorés afin de notifier son importance et d'éviter qu'elle soit supprimée par mégarde. Pour s'assurer que cette image soit utilisée, nous avons empêché le chargement de QR Codes seuls appartenant à une famille.\\
\par
Le chargement d'un QR Code ou d'une famille de QR Codes s'effectue dans la classe \classe{QRCodeLoader} du Contrôleur. Elle instancie des sous-classes de \classe{QRCode} à partir de la racine XML des QR Codes lus dans les métadonnées de l'image, et les renvoie sous forme d'objet unique pour les QR Codes seuls ou sous forme de tableau pour les QR Codes appartenant à une famille.