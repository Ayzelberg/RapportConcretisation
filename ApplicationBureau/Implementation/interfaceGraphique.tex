\par
L'interface graphique offre la possibilité à l'utilisateur d'interagir avec l'application.

\begin{figure}[!h]
	\centering
   \includegraphics[scale=0.25]{img/interface.png}
   \caption{Interface de création d'une famille de qrcodes}
\end{figure}

\par
Elle constituée de trois blocs partant de la gauche vers la droite : 
\begin{enumerate}
\item Bloc 1 : partant de haut en bas, il regroupe les boutons permettant les opérations suivantes :
	\begin{itemize}
	\item la création d'un nouveau projet de QR Code unique ou d'une famille de QR Codes.
	\item le chargement ou l'importation d'un projet de QR Code unique ou d'une famille de QR Codes.
	\item l'exportation ou la sauvegarde d'un projet de QR Code unique ou d'une famille de QR Codes.
	\item l'aperçu de l'image d'une famille de QR Codes.
		\begin{figure}[!h]
			\centering
		   	\includegraphics[scale=0.25]{img/image-famille.png}
		   	\caption{Interface de création d'une famille de qrcodes}
		\end{figure}
	
	L'image de la famille contient le nom de la famille de QR Codes, le nombre de QR Codes existant dans cette famille et la date de création du projet.\\
	Les boutons Sauvegarder et Fermer permettent respectivement de sauvegarder le projet, de fermer l'aperçu.\\
	À noter que les quatres QR Codes figurant sur l'image de la famille, ne sont que des images statiques; elles ne représentent pas des QR Codes en tant que tels.
	
	\item la fermeture d'un projet.
	\end{itemize}
	
\item Bloc 2 : il contient un formulaire qui regroupe des champs (musique ou texte). À coté de chaque champ se trouvent des boutons pour créer, supprimer et lire le contenu du champ par synthèse vocale.\\
À la fin du formulaire, de la gauche vers la droite, se trouvent :
	\begin{itemize}
	\item une case à cocher pour afficher ou masquer les options du texte en braille à savoir le texte et la couleur du texte grâce à une palette de couleur
	\item une palette de couleur pour la couleur du QR Code.
	\end{itemize}
Dans le cas d'un projet de famille de QR Codes, on note l'apparition :
	\begin{itemize}
	\item d'une zone de texte pour le nom de la famille, en haut du formulaire.
	\item d'une liste d'onglets, chacun faisant référence à un formulaire. À coté de chaque onglet, figurent deux boutons pour ajouter un onglet ou supprimer celui actif; chaque onglet représente un QR Code.
	\end{itemize}
	
\item Bloc 3 : partant du haut vers le bas, il contient : 
	\begin{itemize}
	\item un bouton pour prévisualiser un QR Code à partir des informations du formulaire (celui de l'onglet actif dans le cas d'un projet de famille de QR Codes).
	\item un bouton lire pour lire par synthèse vocale tous les champs du formulaire (le formulaire de l'onglet actif dans le cas d'un projet de famille de QR codes.
	\end{itemize}
Dans le cas d'un projet de famille de QR Code, on note l'apparition :
	\begin{itemize}
	\item d'une zone de texte pour le nom de la famille, en haut du formulaire.
	\item d'une liste d'onglets, chacun faisant référence à un formulaire. À coté de chaque onglet, figurent deux boutons pour ajouter un onglet ou supprimer celui actif; chaque onglet représente un QR Code.
		\item d'une image représentant le QR Code généré après la prévisualisation; éventuellement il y a au milieu de celle-ci un code en braille représentant le texte en braille.
	\end{itemize}
\end{enumerate}
