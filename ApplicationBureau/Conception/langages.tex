\par
Le choix des langages de programmation pour l'application de bureau a été discuté lors de la première séance. Nous avons évoqué le couple Java/Swing qui avait l'avantage d'être le plus simple à programmer et qui aurait été en cohérence avec l'application Android, mais qui nécessitait un environnement Java sur les machines des utilisateurs. Nous avons également évoqué le couple C++/Qt qui était le plus portable mais potentiellement trop lourd pour l'utilisation qu'on en aurait. 
\par
Les responsables du projet ont finalement opté pour le framework\footnote{Un framework désigne un ensemble cohérent de composants logiciels structurels, qui sert à créer les fondations ainsi que les grandes lignes de tout ou d’une partie d'un logiciel (Wikipédia).} Electron entre les deux premières séances. Ce framework permet de développer des applications de bureau avec des langages web (Javascript/HTML/CSS), et d'utiliser les modules Node.js. Ce choix a été motivé par l'intérêt qu'avait le Javascript d'être facilement utilisable pour accéder à Google Drive (voir \ref{stockage}).\\

\par
